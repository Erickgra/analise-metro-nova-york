
% Default to the notebook output style

    


% Inherit from the specified cell style.




    
\documentclass[11pt]{article}

    
    
    \usepackage[T1]{fontenc}
    % Nicer default font (+ math font) than Computer Modern for most use cases
    \usepackage{mathpazo}

    % Basic figure setup, for now with no caption control since it's done
    % automatically by Pandoc (which extracts ![](path) syntax from Markdown).
    \usepackage{graphicx}
    % We will generate all images so they have a width \maxwidth. This means
    % that they will get their normal width if they fit onto the page, but
    % are scaled down if they would overflow the margins.
    \makeatletter
    \def\maxwidth{\ifdim\Gin@nat@width>\linewidth\linewidth
    \else\Gin@nat@width\fi}
    \makeatother
    \let\Oldincludegraphics\includegraphics
    % Set max figure width to be 80% of text width, for now hardcoded.
    \renewcommand{\includegraphics}[1]{\Oldincludegraphics[width=.8\maxwidth]{#1}}
    % Ensure that by default, figures have no caption (until we provide a
    % proper Figure object with a Caption API and a way to capture that
    % in the conversion process - todo).
    \usepackage{caption}
    \DeclareCaptionLabelFormat{nolabel}{}
    \captionsetup{labelformat=nolabel}

    \usepackage{adjustbox} % Used to constrain images to a maximum size 
    \usepackage{xcolor} % Allow colors to be defined
    \usepackage{enumerate} % Needed for markdown enumerations to work
    \usepackage{geometry} % Used to adjust the document margins
    \usepackage{amsmath} % Equations
    \usepackage{amssymb} % Equations
    \usepackage{textcomp} % defines textquotesingle
    % Hack from http://tex.stackexchange.com/a/47451/13684:
    \AtBeginDocument{%
        \def\PYZsq{\textquotesingle}% Upright quotes in Pygmentized code
    }
    \usepackage{upquote} % Upright quotes for verbatim code
    \usepackage{eurosym} % defines \euro
    \usepackage[mathletters]{ucs} % Extended unicode (utf-8) support
    \usepackage[utf8x]{inputenc} % Allow utf-8 characters in the tex document
    \usepackage{fancyvrb} % verbatim replacement that allows latex
    \usepackage{grffile} % extends the file name processing of package graphics 
                         % to support a larger range 
    % The hyperref package gives us a pdf with properly built
    % internal navigation ('pdf bookmarks' for the table of contents,
    % internal cross-reference links, web links for URLs, etc.)
    \usepackage{hyperref}
    \usepackage{longtable} % longtable support required by pandoc >1.10
    \usepackage{booktabs}  % table support for pandoc > 1.12.2
    \usepackage[inline]{enumitem} % IRkernel/repr support (it uses the enumerate* environment)
    \usepackage[normalem]{ulem} % ulem is needed to support strikethroughs (\sout)
                                % normalem makes italics be italics, not underlines
    

    
    
    % Colors for the hyperref package
    \definecolor{urlcolor}{rgb}{0,.145,.698}
    \definecolor{linkcolor}{rgb}{.71,0.21,0.01}
    \definecolor{citecolor}{rgb}{.12,.54,.11}

    % ANSI colors
    \definecolor{ansi-black}{HTML}{3E424D}
    \definecolor{ansi-black-intense}{HTML}{282C36}
    \definecolor{ansi-red}{HTML}{E75C58}
    \definecolor{ansi-red-intense}{HTML}{B22B31}
    \definecolor{ansi-green}{HTML}{00A250}
    \definecolor{ansi-green-intense}{HTML}{007427}
    \definecolor{ansi-yellow}{HTML}{DDB62B}
    \definecolor{ansi-yellow-intense}{HTML}{B27D12}
    \definecolor{ansi-blue}{HTML}{208FFB}
    \definecolor{ansi-blue-intense}{HTML}{0065CA}
    \definecolor{ansi-magenta}{HTML}{D160C4}
    \definecolor{ansi-magenta-intense}{HTML}{A03196}
    \definecolor{ansi-cyan}{HTML}{60C6C8}
    \definecolor{ansi-cyan-intense}{HTML}{258F8F}
    \definecolor{ansi-white}{HTML}{C5C1B4}
    \definecolor{ansi-white-intense}{HTML}{A1A6B2}

    % commands and environments needed by pandoc snippets
    % extracted from the output of `pandoc -s`
    \providecommand{\tightlist}{%
      \setlength{\itemsep}{0pt}\setlength{\parskip}{0pt}}
    \DefineVerbatimEnvironment{Highlighting}{Verbatim}{commandchars=\\\{\}}
    % Add ',fontsize=\small' for more characters per line
    \newenvironment{Shaded}{}{}
    \newcommand{\KeywordTok}[1]{\textcolor[rgb]{0.00,0.44,0.13}{\textbf{{#1}}}}
    \newcommand{\DataTypeTok}[1]{\textcolor[rgb]{0.56,0.13,0.00}{{#1}}}
    \newcommand{\DecValTok}[1]{\textcolor[rgb]{0.25,0.63,0.44}{{#1}}}
    \newcommand{\BaseNTok}[1]{\textcolor[rgb]{0.25,0.63,0.44}{{#1}}}
    \newcommand{\FloatTok}[1]{\textcolor[rgb]{0.25,0.63,0.44}{{#1}}}
    \newcommand{\CharTok}[1]{\textcolor[rgb]{0.25,0.44,0.63}{{#1}}}
    \newcommand{\StringTok}[1]{\textcolor[rgb]{0.25,0.44,0.63}{{#1}}}
    \newcommand{\CommentTok}[1]{\textcolor[rgb]{0.38,0.63,0.69}{\textit{{#1}}}}
    \newcommand{\OtherTok}[1]{\textcolor[rgb]{0.00,0.44,0.13}{{#1}}}
    \newcommand{\AlertTok}[1]{\textcolor[rgb]{1.00,0.00,0.00}{\textbf{{#1}}}}
    \newcommand{\FunctionTok}[1]{\textcolor[rgb]{0.02,0.16,0.49}{{#1}}}
    \newcommand{\RegionMarkerTok}[1]{{#1}}
    \newcommand{\ErrorTok}[1]{\textcolor[rgb]{1.00,0.00,0.00}{\textbf{{#1}}}}
    \newcommand{\NormalTok}[1]{{#1}}
    
    % Additional commands for more recent versions of Pandoc
    \newcommand{\ConstantTok}[1]{\textcolor[rgb]{0.53,0.00,0.00}{{#1}}}
    \newcommand{\SpecialCharTok}[1]{\textcolor[rgb]{0.25,0.44,0.63}{{#1}}}
    \newcommand{\VerbatimStringTok}[1]{\textcolor[rgb]{0.25,0.44,0.63}{{#1}}}
    \newcommand{\SpecialStringTok}[1]{\textcolor[rgb]{0.73,0.40,0.53}{{#1}}}
    \newcommand{\ImportTok}[1]{{#1}}
    \newcommand{\DocumentationTok}[1]{\textcolor[rgb]{0.73,0.13,0.13}{\textit{{#1}}}}
    \newcommand{\AnnotationTok}[1]{\textcolor[rgb]{0.38,0.63,0.69}{\textbf{\textit{{#1}}}}}
    \newcommand{\CommentVarTok}[1]{\textcolor[rgb]{0.38,0.63,0.69}{\textbf{\textit{{#1}}}}}
    \newcommand{\VariableTok}[1]{\textcolor[rgb]{0.10,0.09,0.49}{{#1}}}
    \newcommand{\ControlFlowTok}[1]{\textcolor[rgb]{0.00,0.44,0.13}{\textbf{{#1}}}}
    \newcommand{\OperatorTok}[1]{\textcolor[rgb]{0.40,0.40,0.40}{{#1}}}
    \newcommand{\BuiltInTok}[1]{{#1}}
    \newcommand{\ExtensionTok}[1]{{#1}}
    \newcommand{\PreprocessorTok}[1]{\textcolor[rgb]{0.74,0.48,0.00}{{#1}}}
    \newcommand{\AttributeTok}[1]{\textcolor[rgb]{0.49,0.56,0.16}{{#1}}}
    \newcommand{\InformationTok}[1]{\textcolor[rgb]{0.38,0.63,0.69}{\textbf{\textit{{#1}}}}}
    \newcommand{\WarningTok}[1]{\textcolor[rgb]{0.38,0.63,0.69}{\textbf{\textit{{#1}}}}}
    
    
    % Define a nice break command that doesn't care if a line doesn't already
    % exist.
    \def\br{\hspace*{\fill} \\* }
    % Math Jax compatability definitions
    \def\gt{>}
    \def\lt{<}
    % Document parameters
    \title{analyzing-subway-data-ndfdsi}
    
    
    

    % Pygments definitions
    
\makeatletter
\def\PY@reset{\let\PY@it=\relax \let\PY@bf=\relax%
    \let\PY@ul=\relax \let\PY@tc=\relax%
    \let\PY@bc=\relax \let\PY@ff=\relax}
\def\PY@tok#1{\csname PY@tok@#1\endcsname}
\def\PY@toks#1+{\ifx\relax#1\empty\else%
    \PY@tok{#1}\expandafter\PY@toks\fi}
\def\PY@do#1{\PY@bc{\PY@tc{\PY@ul{%
    \PY@it{\PY@bf{\PY@ff{#1}}}}}}}
\def\PY#1#2{\PY@reset\PY@toks#1+\relax+\PY@do{#2}}

\expandafter\def\csname PY@tok@w\endcsname{\def\PY@tc##1{\textcolor[rgb]{0.73,0.73,0.73}{##1}}}
\expandafter\def\csname PY@tok@c\endcsname{\let\PY@it=\textit\def\PY@tc##1{\textcolor[rgb]{0.25,0.50,0.50}{##1}}}
\expandafter\def\csname PY@tok@cp\endcsname{\def\PY@tc##1{\textcolor[rgb]{0.74,0.48,0.00}{##1}}}
\expandafter\def\csname PY@tok@k\endcsname{\let\PY@bf=\textbf\def\PY@tc##1{\textcolor[rgb]{0.00,0.50,0.00}{##1}}}
\expandafter\def\csname PY@tok@kp\endcsname{\def\PY@tc##1{\textcolor[rgb]{0.00,0.50,0.00}{##1}}}
\expandafter\def\csname PY@tok@kt\endcsname{\def\PY@tc##1{\textcolor[rgb]{0.69,0.00,0.25}{##1}}}
\expandafter\def\csname PY@tok@o\endcsname{\def\PY@tc##1{\textcolor[rgb]{0.40,0.40,0.40}{##1}}}
\expandafter\def\csname PY@tok@ow\endcsname{\let\PY@bf=\textbf\def\PY@tc##1{\textcolor[rgb]{0.67,0.13,1.00}{##1}}}
\expandafter\def\csname PY@tok@nb\endcsname{\def\PY@tc##1{\textcolor[rgb]{0.00,0.50,0.00}{##1}}}
\expandafter\def\csname PY@tok@nf\endcsname{\def\PY@tc##1{\textcolor[rgb]{0.00,0.00,1.00}{##1}}}
\expandafter\def\csname PY@tok@nc\endcsname{\let\PY@bf=\textbf\def\PY@tc##1{\textcolor[rgb]{0.00,0.00,1.00}{##1}}}
\expandafter\def\csname PY@tok@nn\endcsname{\let\PY@bf=\textbf\def\PY@tc##1{\textcolor[rgb]{0.00,0.00,1.00}{##1}}}
\expandafter\def\csname PY@tok@ne\endcsname{\let\PY@bf=\textbf\def\PY@tc##1{\textcolor[rgb]{0.82,0.25,0.23}{##1}}}
\expandafter\def\csname PY@tok@nv\endcsname{\def\PY@tc##1{\textcolor[rgb]{0.10,0.09,0.49}{##1}}}
\expandafter\def\csname PY@tok@no\endcsname{\def\PY@tc##1{\textcolor[rgb]{0.53,0.00,0.00}{##1}}}
\expandafter\def\csname PY@tok@nl\endcsname{\def\PY@tc##1{\textcolor[rgb]{0.63,0.63,0.00}{##1}}}
\expandafter\def\csname PY@tok@ni\endcsname{\let\PY@bf=\textbf\def\PY@tc##1{\textcolor[rgb]{0.60,0.60,0.60}{##1}}}
\expandafter\def\csname PY@tok@na\endcsname{\def\PY@tc##1{\textcolor[rgb]{0.49,0.56,0.16}{##1}}}
\expandafter\def\csname PY@tok@nt\endcsname{\let\PY@bf=\textbf\def\PY@tc##1{\textcolor[rgb]{0.00,0.50,0.00}{##1}}}
\expandafter\def\csname PY@tok@nd\endcsname{\def\PY@tc##1{\textcolor[rgb]{0.67,0.13,1.00}{##1}}}
\expandafter\def\csname PY@tok@s\endcsname{\def\PY@tc##1{\textcolor[rgb]{0.73,0.13,0.13}{##1}}}
\expandafter\def\csname PY@tok@sd\endcsname{\let\PY@it=\textit\def\PY@tc##1{\textcolor[rgb]{0.73,0.13,0.13}{##1}}}
\expandafter\def\csname PY@tok@si\endcsname{\let\PY@bf=\textbf\def\PY@tc##1{\textcolor[rgb]{0.73,0.40,0.53}{##1}}}
\expandafter\def\csname PY@tok@se\endcsname{\let\PY@bf=\textbf\def\PY@tc##1{\textcolor[rgb]{0.73,0.40,0.13}{##1}}}
\expandafter\def\csname PY@tok@sr\endcsname{\def\PY@tc##1{\textcolor[rgb]{0.73,0.40,0.53}{##1}}}
\expandafter\def\csname PY@tok@ss\endcsname{\def\PY@tc##1{\textcolor[rgb]{0.10,0.09,0.49}{##1}}}
\expandafter\def\csname PY@tok@sx\endcsname{\def\PY@tc##1{\textcolor[rgb]{0.00,0.50,0.00}{##1}}}
\expandafter\def\csname PY@tok@m\endcsname{\def\PY@tc##1{\textcolor[rgb]{0.40,0.40,0.40}{##1}}}
\expandafter\def\csname PY@tok@gh\endcsname{\let\PY@bf=\textbf\def\PY@tc##1{\textcolor[rgb]{0.00,0.00,0.50}{##1}}}
\expandafter\def\csname PY@tok@gu\endcsname{\let\PY@bf=\textbf\def\PY@tc##1{\textcolor[rgb]{0.50,0.00,0.50}{##1}}}
\expandafter\def\csname PY@tok@gd\endcsname{\def\PY@tc##1{\textcolor[rgb]{0.63,0.00,0.00}{##1}}}
\expandafter\def\csname PY@tok@gi\endcsname{\def\PY@tc##1{\textcolor[rgb]{0.00,0.63,0.00}{##1}}}
\expandafter\def\csname PY@tok@gr\endcsname{\def\PY@tc##1{\textcolor[rgb]{1.00,0.00,0.00}{##1}}}
\expandafter\def\csname PY@tok@ge\endcsname{\let\PY@it=\textit}
\expandafter\def\csname PY@tok@gs\endcsname{\let\PY@bf=\textbf}
\expandafter\def\csname PY@tok@gp\endcsname{\let\PY@bf=\textbf\def\PY@tc##1{\textcolor[rgb]{0.00,0.00,0.50}{##1}}}
\expandafter\def\csname PY@tok@go\endcsname{\def\PY@tc##1{\textcolor[rgb]{0.53,0.53,0.53}{##1}}}
\expandafter\def\csname PY@tok@gt\endcsname{\def\PY@tc##1{\textcolor[rgb]{0.00,0.27,0.87}{##1}}}
\expandafter\def\csname PY@tok@err\endcsname{\def\PY@bc##1{\setlength{\fboxsep}{0pt}\fcolorbox[rgb]{1.00,0.00,0.00}{1,1,1}{\strut ##1}}}
\expandafter\def\csname PY@tok@kc\endcsname{\let\PY@bf=\textbf\def\PY@tc##1{\textcolor[rgb]{0.00,0.50,0.00}{##1}}}
\expandafter\def\csname PY@tok@kd\endcsname{\let\PY@bf=\textbf\def\PY@tc##1{\textcolor[rgb]{0.00,0.50,0.00}{##1}}}
\expandafter\def\csname PY@tok@kn\endcsname{\let\PY@bf=\textbf\def\PY@tc##1{\textcolor[rgb]{0.00,0.50,0.00}{##1}}}
\expandafter\def\csname PY@tok@kr\endcsname{\let\PY@bf=\textbf\def\PY@tc##1{\textcolor[rgb]{0.00,0.50,0.00}{##1}}}
\expandafter\def\csname PY@tok@bp\endcsname{\def\PY@tc##1{\textcolor[rgb]{0.00,0.50,0.00}{##1}}}
\expandafter\def\csname PY@tok@fm\endcsname{\def\PY@tc##1{\textcolor[rgb]{0.00,0.00,1.00}{##1}}}
\expandafter\def\csname PY@tok@vc\endcsname{\def\PY@tc##1{\textcolor[rgb]{0.10,0.09,0.49}{##1}}}
\expandafter\def\csname PY@tok@vg\endcsname{\def\PY@tc##1{\textcolor[rgb]{0.10,0.09,0.49}{##1}}}
\expandafter\def\csname PY@tok@vi\endcsname{\def\PY@tc##1{\textcolor[rgb]{0.10,0.09,0.49}{##1}}}
\expandafter\def\csname PY@tok@vm\endcsname{\def\PY@tc##1{\textcolor[rgb]{0.10,0.09,0.49}{##1}}}
\expandafter\def\csname PY@tok@sa\endcsname{\def\PY@tc##1{\textcolor[rgb]{0.73,0.13,0.13}{##1}}}
\expandafter\def\csname PY@tok@sb\endcsname{\def\PY@tc##1{\textcolor[rgb]{0.73,0.13,0.13}{##1}}}
\expandafter\def\csname PY@tok@sc\endcsname{\def\PY@tc##1{\textcolor[rgb]{0.73,0.13,0.13}{##1}}}
\expandafter\def\csname PY@tok@dl\endcsname{\def\PY@tc##1{\textcolor[rgb]{0.73,0.13,0.13}{##1}}}
\expandafter\def\csname PY@tok@s2\endcsname{\def\PY@tc##1{\textcolor[rgb]{0.73,0.13,0.13}{##1}}}
\expandafter\def\csname PY@tok@sh\endcsname{\def\PY@tc##1{\textcolor[rgb]{0.73,0.13,0.13}{##1}}}
\expandafter\def\csname PY@tok@s1\endcsname{\def\PY@tc##1{\textcolor[rgb]{0.73,0.13,0.13}{##1}}}
\expandafter\def\csname PY@tok@mb\endcsname{\def\PY@tc##1{\textcolor[rgb]{0.40,0.40,0.40}{##1}}}
\expandafter\def\csname PY@tok@mf\endcsname{\def\PY@tc##1{\textcolor[rgb]{0.40,0.40,0.40}{##1}}}
\expandafter\def\csname PY@tok@mh\endcsname{\def\PY@tc##1{\textcolor[rgb]{0.40,0.40,0.40}{##1}}}
\expandafter\def\csname PY@tok@mi\endcsname{\def\PY@tc##1{\textcolor[rgb]{0.40,0.40,0.40}{##1}}}
\expandafter\def\csname PY@tok@il\endcsname{\def\PY@tc##1{\textcolor[rgb]{0.40,0.40,0.40}{##1}}}
\expandafter\def\csname PY@tok@mo\endcsname{\def\PY@tc##1{\textcolor[rgb]{0.40,0.40,0.40}{##1}}}
\expandafter\def\csname PY@tok@ch\endcsname{\let\PY@it=\textit\def\PY@tc##1{\textcolor[rgb]{0.25,0.50,0.50}{##1}}}
\expandafter\def\csname PY@tok@cm\endcsname{\let\PY@it=\textit\def\PY@tc##1{\textcolor[rgb]{0.25,0.50,0.50}{##1}}}
\expandafter\def\csname PY@tok@cpf\endcsname{\let\PY@it=\textit\def\PY@tc##1{\textcolor[rgb]{0.25,0.50,0.50}{##1}}}
\expandafter\def\csname PY@tok@c1\endcsname{\let\PY@it=\textit\def\PY@tc##1{\textcolor[rgb]{0.25,0.50,0.50}{##1}}}
\expandafter\def\csname PY@tok@cs\endcsname{\let\PY@it=\textit\def\PY@tc##1{\textcolor[rgb]{0.25,0.50,0.50}{##1}}}

\def\PYZbs{\char`\\}
\def\PYZus{\char`\_}
\def\PYZob{\char`\{}
\def\PYZcb{\char`\}}
\def\PYZca{\char`\^}
\def\PYZam{\char`\&}
\def\PYZlt{\char`\<}
\def\PYZgt{\char`\>}
\def\PYZsh{\char`\#}
\def\PYZpc{\char`\%}
\def\PYZdl{\char`\$}
\def\PYZhy{\char`\-}
\def\PYZsq{\char`\'}
\def\PYZdq{\char`\"}
\def\PYZti{\char`\~}
% for compatibility with earlier versions
\def\PYZat{@}
\def\PYZlb{[}
\def\PYZrb{]}
\makeatother


    % Exact colors from NB
    \definecolor{incolor}{rgb}{0.0, 0.0, 0.5}
    \definecolor{outcolor}{rgb}{0.545, 0.0, 0.0}



    
    % Prevent overflowing lines due to hard-to-break entities
    \sloppy 
    % Setup hyperref package
    \hypersetup{
      breaklinks=true,  % so long urls are correctly broken across lines
      colorlinks=true,
      urlcolor=urlcolor,
      linkcolor=linkcolor,
      citecolor=citecolor,
      }
    % Slightly bigger margins than the latex defaults
    
    \geometry{verbose,tmargin=1in,bmargin=1in,lmargin=1in,rmargin=1in}
    
    

    \begin{document}
    
    
    \maketitle
    
    

    
    \section{Subway Data Analysis}\label{subway-data-analysis}

\subsection{Introduction}\label{introduction}

O sistema de ônibus e trens de Nova Iorque - o Metro Transit Authority -
fornece seus dados para download através de arquivos csv. Uma das
informações disponíveis são os dados das catracas do metrô que contém
logs semanais de entradas cumulativas e saídas por catraca por estação
de metrô em algum intervalo de tempo.

Neste projeto iremos utilizar apenas os das catraca disponíveis em:
http://web.mta.info/developers/turnstile.html.

    \section{Sobre este projeto}\label{sobre-este-projeto}

Neste projeto você irá aplicar todos os conhecimentos adquiridos neste
primeiro mês de curso. Iremos praticar tarefas básicas de aquisição,
limpeza de dados e nesse processo iremos descobrir coisas essenciais
sobre os dados utilizando o que foi aprendido no curso de estatística.

O objetivo deste projeto é explorar a relação entre os dados das
catracas do metro de Nova Iorque e o clima no dia da coleta. Para isso,
além dos dados do metrô, precisaremos os dados de clima da cidade de
Nova Iorque.

Os principais pontos que serão verificados neste trabalho:

\begin{itemize}
\tightlist
\item
  Coleta de dados da internet
\item
  Utilização de estatística para análise de dados
\item
  Manipulação de dados e criação de gráficos simples com o
  \texttt{Pandas}
\end{itemize}

\emph{Como conseguir ajuda}: Sugerimos que tente os seguintes canais,
nas seguintes ordens:

\begin{longtable}[]{@{}lllll@{}}
\toprule
Tipo de dúvida\Canais & Google & Fórum & Slack & Email\tabularnewline
\midrule
\endhead
Programação Pyhon e Pandas & 1 & 2 & 3 &\tabularnewline
Requisitos do projeto & & 1 & 2 & 3\tabularnewline
Partes específicas do Projeto & & 1 & 2 & 3\tabularnewline
\bottomrule
\end{longtable}

Os endereços dos canais são:

\begin{itemize}
\tightlist
\item
  Fórum: https://discussions.udacity.com/c/ndfdsi-project
\item
  Slack:
  \href{https://udacity-br.slack.com/messages/C5MT6E3E1}{udacity-br.slack.com}
\item
  Email: data-suporte@udacity.com
\end{itemize}

\textbf{Espera-se que o estudante entregue este relatório com:}

\begin{itemize}
\tightlist
\item
  Todos os TODO feitos, pois eles são essenciais para que o código rode
  corretamente
\item
  O arquivo ipynb exportado como html
\end{itemize}

Para entregar este projeto, vá a
\href{https://coco.udacity.com/nanodegrees/nd111/locale/pt-br/versions/1.0.0/parts/339726/modules/339733/lessons/340886/project}{sala
de aula} e submeta o seu \texttt{.ipynb} e o html, zipados.

    \section{Lembretes}\label{lembretes}

Antes de começarmos, alguns lembretes devem ter em mente ao usar os
notebooks iPython:

\begin{itemize}
\tightlist
\item
  Lembre-se de que você pode ver do lado esquerdo de uma célula de
  código quando foi executado pela última vez se houver um número dentro
  das chaves.
\item
  Quando você inicia uma nova sessão do notebook, certifique-se de
  executar todas as células até o ponto em que você deixou a última vez.
  Mesmo que a saída ainda seja visível a partir de quando você executou
  as células em sua sessão anterior, o kernel começa em um estado novo,
  então você precisará recarregar os dados, etc. em uma nova sessão.
\item
  O ponto anterior é útil para ter em mente se suas respostas não
  correspondem ao que é esperado nos questionários da aula. Tente
  recarregar os dados e execute todas as etapas de processamento um a um
  para garantir que você esteja trabalhando com as mesmas variáveis e
  dados que estão em cada fase do questionário.
\end{itemize}

    \subsection{Seção 1 - Coleta de
Dados}\label{seuxe7uxe3o-1---coleta-de-dados}

\subsubsection{\texorpdfstring{\emph{Exercicio
1.1}}{Exercicio 1.1}}\label{exercicio-1.1}

Mãos a obra!! Agora é sua vez de coletar os dados. Escreva abaixo um
código python que acesse o link
http://web.mta.info/developers/turnstile.html e baixe os arquivos do mês
de junho de 2017. O arquivo deverá ser salvo com o nome
turnstile\_100617.txt onde 10/06/17 é a data do arquivo.

Abaixo seguem alguns comandos que poderão te ajudar:

Utilize a biblioteca \textbf{urllib} para abrir e resgatar uma página da
web. Utilize o comando abaixo onde \textbf{url} será o caminho da página
da web onde se encontra o arquivo:

\begin{Shaded}
\begin{Highlighting}[]
\NormalTok{u }\OperatorTok{=}\NormalTok{ urllib.urlopen(url)}
\NormalTok{html }\OperatorTok{=}\NormalTok{ u.read()}
\end{Highlighting}
\end{Shaded}

Utilize a biblioteca \textbf{BeautifulSoup} para procurar na página pelo
link do arquivo que deseja baixar. Utilize o comando abaixo para criar o
seu objeto \emph{soup} e procurar por todas as tags 'a'no documento:

\begin{Shaded}
\begin{Highlighting}[]
\NormalTok{soup }\OperatorTok{=}\NormalTok{ BeautifulSoup(html, }\StringTok{"html.parser"}\NormalTok{)}
\NormalTok{links }\OperatorTok{=}\NormalTok{ soup.find_all(}\StringTok{'a'}\NormalTok{)}
\end{Highlighting}
\end{Shaded}

Uma dica para baixar apenas os arquivos do mês de junho é verificar a
data no nome do arquivo. Por exemplo, para baixar o arquivo do dia
17/06/2017 verifique se o link termina com
\emph{"turnstile\_170610.txt"}. Se não fizer isso você baixará todos os
arquivos da página. Para fazer isso utilize o comando conforme abaixo:

\begin{Shaded}
\begin{Highlighting}[]
\ControlFlowTok{if} \StringTok{'1706'} \KeywordTok{in}\NormalTok{ link.get(}\StringTok{'href'}\NormalTok{):}
\end{Highlighting}
\end{Shaded}

E a dica final é utilizar o comando abaixo para fazer o download do
arquivo txt:

\begin{Shaded}
\begin{Highlighting}[]
\NormalTok{urllib.urlretrieve(link_do_arquivo, filename)}
\end{Highlighting}
\end{Shaded}

Lembre-se, primeiro, carregue todos os pacotes e funções que você estará
usando em sua análise.

    \begin{Verbatim}[commandchars=\\\{\}]
{\color{incolor}In [{\color{incolor}22}]:} \PY{k+kn}{import} \PY{n+nn}{urllib}
         \PY{k+kn}{from} \PY{n+nn}{bs4} \PY{k}{import} \PY{n}{BeautifulSoup}
         \PY{k+kn}{from} \PY{n+nn}{urllib}\PY{n+nn}{.}\PY{n+nn}{request} \PY{k}{import} \PY{n}{urlopen}\PY{p}{,} \PY{n}{urlretrieve}
         \PY{c+c1}{\PYZsh{}Abre a página e ler o conteúdo html}
         \PY{n}{maingPageHtml} \PY{o}{=} \PY{n}{urllib}\PY{o}{.}\PY{n}{request}\PY{o}{.}\PY{n}{urlopen}\PY{p}{(}\PY{l+s+s2}{\PYZdq{}}\PY{l+s+s2}{http://web.mta.info/developers/turnstile.html}\PY{l+s+s2}{\PYZdq{}} \PY{p}{)}\PY{o}{.}\PY{n}{read}\PY{p}{(}\PY{p}{)}
         \PY{n}{soup} \PY{o}{=} \PY{n}{BeautifulSoup}\PY{p}{(}\PY{n}{maingPageHtml}\PY{p}{,} \PY{l+s+s2}{\PYZdq{}}\PY{l+s+s2}{html.parser}\PY{l+s+s2}{\PYZdq{}}\PY{p}{)}
         \PY{c+c1}{\PYZsh{}obtém todos os links da página}
         \PY{n}{links} \PY{o}{=} \PY{n}{soup}\PY{o}{.}\PY{n}{find\PYZus{}all}\PY{p}{(}\PY{l+s+s1}{\PYZsq{}}\PY{l+s+s1}{a}\PY{l+s+s1}{\PYZsq{}}\PY{p}{)}
         \PY{n}{totalArquivos} \PY{o}{=} \PY{l+m+mi}{0}
         \PY{k}{for} \PY{n}{link} \PY{o+ow}{in} \PY{n}{links}\PY{p}{:}
             \PY{n}{href}\PY{o}{=} \PY{n}{link}\PY{o}{.}\PY{n}{get}\PY{p}{(}\PY{l+s+s1}{\PYZsq{}}\PY{l+s+s1}{href}\PY{l+s+s1}{\PYZsq{}}\PY{p}{)}
             \PY{k}{if} \PY{n}{href} \PY{o}{!=} \PY{k+kc}{None} \PY{o+ow}{and} \PY{l+s+s1}{\PYZsq{}}\PY{l+s+s1}{1706}\PY{l+s+s1}{\PYZsq{}} \PY{o+ow}{in} \PY{n}{href}\PY{p}{:}
                 \PY{n}{totalArquivos} \PY{o}{+}\PY{o}{=} \PY{l+m+mi}{1}
                 \PY{c+c1}{\PYZsh{}recupera o nome do arquivo}
                 \PY{n}{filename} \PY{o}{=} \PY{n}{href}\PY{o}{.}\PY{n}{rsplit}\PY{p}{(}\PY{l+s+s1}{\PYZsq{}}\PY{l+s+s1}{/}\PY{l+s+s1}{\PYZsq{}}\PY{p}{,} \PY{l+m+mi}{1}\PY{p}{)}\PY{p}{[}\PY{o}{\PYZhy{}}\PY{l+m+mi}{1}\PY{p}{]}
                 \PY{n}{url} \PY{o}{=} \PY{l+s+s1}{\PYZsq{}}\PY{l+s+s1}{http://web.mta.info/developers/}\PY{l+s+s1}{\PYZsq{}} \PY{o}{+} \PY{n}{href}
                 \PY{n+nb}{print}\PY{p}{(}\PY{l+s+s2}{\PYZdq{}}\PY{l+s+s2}{Iniciando o download do arquivo }\PY{l+s+si}{\PYZpc{}s}\PY{l+s+s2}{\PYZdq{}} \PY{o}{\PYZpc{}}\PY{p}{(}\PY{n}{filename}\PY{p}{)}\PY{p}{)}
                 \PY{n}{urlretrieve}\PY{p}{(}\PY{n}{url}\PY{p}{,} \PY{n}{filename}\PY{p}{)}
                 
         \PY{n+nb}{print}\PY{p}{(}\PY{l+s+s2}{\PYZdq{}}\PY{l+s+s2}{Foram baixados: }\PY{l+s+si}{\PYZpc{}s}\PY{l+s+s2}{\PYZdq{}} \PY{o}{\PYZpc{}}\PY{p}{(}\PY{n}{totalArquivos}\PY{p}{)}\PY{p}{)}
\end{Verbatim}


    \begin{Verbatim}[commandchars=\\\{\}]
http://web.mta.info/developers/data/nyct/turnstile/turnstile\_170624.txt
Iniciando o download do arquivo turnstile\_170624.txt
http://web.mta.info/developers/data/nyct/turnstile/turnstile\_170617.txt
Iniciando o download do arquivo turnstile\_170617.txt
http://web.mta.info/developers/data/nyct/turnstile/turnstile\_170610.txt
Iniciando o download do arquivo turnstile\_170610.txt
http://web.mta.info/developers/data/nyct/turnstile/turnstile\_170603.txt
Iniciando o download do arquivo turnstile\_170603.txt
Foram encontrados 4

    \end{Verbatim}

    \subsubsection{\texorpdfstring{\emph{Exercicio
1.2}}{Exercicio 1.2}}\label{exercicio-1.2}

Escreva uma função que pegue a lista de nomes dos arquivos que você
baixou no exercicio 1.1 e consolide-os em um único arquivo. Deve existir
apenas uma linha de cabeçalho no arquivo de saida.

Por exemplo, se o arquivo\_1 tiver: linha 1... linha 2...

e o outro arquivo, arquivo\_2 tiver: linha 3... linha 4... linha 5...

Devemos combinar o arquivo\_1 com arquivo\_2 em um arquivo mestre
conforme abaixo:

'C/A, UNIT, SCP, DATEn, TIMEn, DESCn, ENTRIESn, EXITSn' linha 1... linha
2... linha 3... linha 4... linha 5...

    \begin{Verbatim}[commandchars=\\\{\}]
{\color{incolor}In [{\color{incolor}29}]:} \PY{k}{def} \PY{n+nf}{create\PYZus{}master\PYZus{}turnstile\PYZus{}file}\PY{p}{(}\PY{n}{filenames}\PY{p}{,} \PY{n}{output\PYZus{}file}\PY{p}{)}\PY{p}{:}
             \PY{k}{with} \PY{n+nb}{open}\PY{p}{(}\PY{n}{output\PYZus{}file}\PY{p}{,} \PY{l+s+s1}{\PYZsq{}}\PY{l+s+s1}{w}\PY{l+s+s1}{\PYZsq{}}\PY{p}{)} \PY{k}{as} \PY{n}{master\PYZus{}file}\PY{p}{:}
                 \PY{n}{master\PYZus{}file}\PY{o}{.}\PY{n}{write}\PY{p}{(}\PY{l+s+s1}{\PYZsq{}}\PY{l+s+s1}{C/A,UNIT,SCP,STATION, LINENAME, DIVISION, DATEn,TIMEn,DESCn,ENTRIESn,EXITSn}\PY{l+s+se}{\PYZbs{}n}\PY{l+s+s1}{\PYZsq{}}\PY{p}{)}
                 \PY{k}{for} \PY{n}{filename} \PY{o+ow}{in} \PY{n}{filenames}\PY{p}{:}
                     \PY{k}{with} \PY{n+nb}{open}\PY{p}{(}\PY{n}{filename}\PY{p}{,} \PY{l+s+s1}{\PYZsq{}}\PY{l+s+s1}{r}\PY{l+s+s1}{\PYZsq{}}\PY{p}{)} \PY{k}{as} \PY{n}{r}\PY{p}{:}
                         \PY{k}{for} \PY{n}{line} \PY{o+ow}{in} \PY{n}{r}\PY{p}{:}
                             \PY{k}{if} \PY{p}{(}\PY{l+s+s1}{\PYZsq{}}\PY{l+s+s1}{C/A,UNIT,SCP,STATION,LINENAME,DIVISION,DATE,TIME,DESC,ENTRIES,EXITS}\PY{l+s+s1}{\PYZsq{}}\PY{p}{)} \PY{o+ow}{in} \PY{n}{line}\PY{p}{:}
                                 \PY{k}{continue}
                             \PY{n}{master\PYZus{}file}\PY{o}{.}\PY{n}{write}\PY{p}{(}\PY{n}{line} \PY{o}{+} \PY{l+s+s1}{\PYZsq{}}\PY{l+s+se}{\PYZbs{}n}\PY{l+s+s1}{\PYZsq{}}\PY{p}{)}    
          
         \PY{n}{filenames} \PY{o}{=} \PY{p}{[}\PY{l+s+s2}{\PYZdq{}}\PY{l+s+s2}{turnstile\PYZus{}170603.txt}\PY{l+s+s2}{\PYZdq{}}\PY{p}{,} \PY{l+s+s2}{\PYZdq{}}\PY{l+s+s2}{turnstile\PYZus{}170610.txt}\PY{l+s+s2}{\PYZdq{}}\PY{p}{,} \PY{l+s+s2}{\PYZdq{}}\PY{l+s+s2}{turnstile\PYZus{}170617.txt}\PY{l+s+s2}{\PYZdq{}}\PY{p}{,} \PY{l+s+s2}{\PYZdq{}}\PY{l+s+s2}{turnstile\PYZus{}170624.txt}\PY{l+s+s2}{\PYZdq{}}\PY{p}{]}
         \PY{n}{output\PYZus{}file} \PY{o}{=} \PY{l+s+s2}{\PYZdq{}}\PY{l+s+s2}{turnstile.txt}\PY{l+s+s2}{\PYZdq{}}
         \PY{n}{create\PYZus{}master\PYZus{}turnstile\PYZus{}file}\PY{p}{(}\PY{n}{filenames}\PY{p}{,} \PY{n}{output\PYZus{}file}\PY{p}{)}            
\end{Verbatim}


    \subsubsection{\texorpdfstring{\emph{Exercicio
1.3}}{Exercicio 1.3}}\label{exercicio-1.3}

Neste exercício, escreva um função que leia o master\_file criado no
exercicio anterior e carregue-o em um pandas dataframe. Esta função deve
filtrar para que o dataframe possua apenas linhas onde a coluna "DESCn"
possua o valor "Regular".

Por exemplo, se o data frame do pandas estiver conforme abaixo:

\begin{verbatim}
,C/A,UNIT,SCP,DATEn,TIMEn,DESCn,ENTRIESn,EXITSn
0,A002,R051,02-00-00,05-01-11,00:00:00,REGULAR,3144312,1088151
1,A002,R051,02-00-00,05-01-11,04:00:00,DOOR,3144335,1088159
2,A002,R051,02-00-00,05-01-11,08:00:00,REGULAR,3144353,1088177
3,A002,R051,02-00-00,05-01-11,12:00:00,DOOR,3144424,1088231
\end{verbatim}

O dataframe deverá ficar conforme abaixo depois de filtrar apenas as
linhas onde a coluna DESCn possua o valor REGULAR:

\begin{verbatim}
0,A002,R051,02-00-00,05-01-11,00:00:00,REGULAR,3144312,1088151
2,A002,R051,02-00-00,05-01-11,08:00:00,REGULAR,3144353,1088177
\end{verbatim}

    \begin{Verbatim}[commandchars=\\\{\}]
{\color{incolor}In [{\color{incolor}32}]:} \PY{k+kn}{import} \PY{n+nn}{pandas} \PY{k}{as} \PY{n+nn}{pd}
         
         \PY{k}{def} \PY{n+nf}{filter\PYZus{}by\PYZus{}regular}\PY{p}{(}\PY{n}{filename}\PY{p}{)}\PY{p}{:}
             \PY{c+c1}{\PYZsh{}abre o arquivo como data frame do pandas}
             \PY{n}{turnstile\PYZus{}data} \PY{o}{=} \PY{n}{pd}\PY{o}{.}\PY{n}{read\PYZus{}csv}\PY{p}{(}\PY{n}{filename}\PY{p}{)}
             \PY{c+c1}{\PYZsh{}filtra para trazer somente os dados que tenham descn igual a regular}
             \PY{n}{turnstile\PYZus{}data} \PY{o}{=} \PY{n}{turnstile\PYZus{}data}\PY{p}{[}\PY{n}{turnstile\PYZus{}data}\PY{p}{[}\PY{l+s+s1}{\PYZsq{}}\PY{l+s+s1}{DESCn}\PY{l+s+s1}{\PYZsq{}}\PY{p}{]} \PY{o}{==} \PY{l+s+s1}{\PYZsq{}}\PY{l+s+s1}{REGULAR}\PY{l+s+s1}{\PYZsq{}}\PY{p}{]}
             \PY{k}{return} \PY{n}{turnstile\PYZus{}data}
         
         \PY{n}{turnstile\PYZus{}regular} \PY{o}{=} \PY{n}{filter\PYZus{}by\PYZus{}regular}\PY{p}{(}\PY{l+s+s1}{\PYZsq{}}\PY{l+s+s1}{turnstile.txt}\PY{l+s+s1}{\PYZsq{}}\PY{p}{)}
         \PY{n}{turnstile\PYZus{}regular}\PY{o}{.}\PY{n}{head}\PY{p}{(}\PY{p}{)}
\end{Verbatim}


\begin{Verbatim}[commandchars=\\\{\}]
{\color{outcolor}Out[{\color{outcolor}32}]:}     C/A  UNIT       SCP STATION  LINENAME  DIVISION       DATEn     TIMEn  \textbackslash{}
         0  A002  R051  02-00-00   59 ST   NQR456W       BMT  05/27/2017  00:00:00   
         1  A002  R051  02-00-00   59 ST   NQR456W       BMT  05/27/2017  04:00:00   
         2  A002  R051  02-00-00   59 ST   NQR456W       BMT  05/27/2017  08:00:00   
         3  A002  R051  02-00-00   59 ST   NQR456W       BMT  05/27/2017  12:00:00   
         4  A002  R051  02-00-00   59 ST   NQR456W       BMT  05/27/2017  16:00:00   
         
              DESCn  ENTRIESn   EXITSn  
         0  REGULAR   6195217  2098317  
         1  REGULAR   6195240  2098318  
         2  REGULAR   6195256  2098347  
         3  REGULAR   6195346  2098432  
         4  REGULAR   6195518  2098491  
\end{Verbatim}
            
    \subsubsection{\texorpdfstring{\emph{Exercicio
1.4}}{Exercicio 1.4}}\label{exercicio-1.4}

Os dados do metrô de NY possui dados cumulativos de entradas e saidas
por linha. Assuma que você possui um dataframe chamado df que contém
apenas linhas para uma catraca em particular (unico SCP, C/A, e UNIT). A
função abaixo deve alterar essas entradas cumulativas para a contagem de
entradas desde a última leitura (entradas desde a última linha do
dataframe).

Mais especificamente, você deverá fazer duas coisas:

1 - Criar uma nova coluna chamada ENTRIESn\_hourly 2 - Inserir nessa
coluna a diferença entre ENTRIESn da coluna atual e a da coluna
anterior. Se a linha possuir alguma NAN, preencha/substitua por 1.

Dica: as funções do pandas shift() e fillna() pode ser úteis nesse
exercicio.

Abaixo tem um exemplo de como seu dataframe deve ficar ao final desse
exercicio:

\begin{verbatim}
    C/A  UNIT       SCP     DATEn     TIMEn    DESCn  ENTRIESn    EXITSn  ENTRIESn_hourly
0     A002  R051  02-00-00  05-01-11  00:00:00  REGULAR   3144312   1088151                1
1     A002  R051  02-00-00  05-01-11  04:00:00  REGULAR   3144335   1088159               23
2     A002  R051  02-00-00  05-01-11  08:00:00  REGULAR   3144353   1088177               18
3     A002  R051  02-00-00  05-01-11  12:00:00  REGULAR   3144424   1088231               71
4     A002  R051  02-00-00  05-01-11  16:00:00  REGULAR   3144594   1088275              170
5     A002  R051  02-00-00  05-01-11  20:00:00  REGULAR   3144808   1088317              214
6     A002  R051  02-00-00  05-02-11  00:00:00  REGULAR   3144895   1088328               87
7     A002  R051  02-00-00  05-02-11  04:00:00  REGULAR   3144905   1088331               10
8     A002  R051  02-00-00  05-02-11  08:00:00  REGULAR   3144941   1088420               36
9     A002  R051  02-00-00  05-02-11  12:00:00  REGULAR   3145094   1088753              153
10    A002  R051  02-00-00  05-02-11  16:00:00  REGULAR   3145337   1088823              243
\end{verbatim}

    \begin{Verbatim}[commandchars=\\\{\}]
{\color{incolor}In [{\color{incolor}42}]:} \PY{k+kn}{import} \PY{n+nn}{pandas}
         \PY{k+kn}{import} \PY{n+nn}{numpy} \PY{k}{as} \PY{n+nn}{np}
         
         \PY{k}{def} \PY{n+nf}{get\PYZus{}hourly\PYZus{}entries}\PY{p}{(}\PY{n}{df}\PY{p}{)}\PY{p}{:}
             
             
             \PY{c+c1}{\PYZsh{}cria uma coluna nova (ENTRIESn\PYZus{}hourly) no dataframe e preenche suas linhas com a diferença de entradas}
             \PY{c+c1}{\PYZsh{} da linha atual com a linha anterior da coluna (ENTRIESn)}
             \PY{n}{df}\PY{p}{[}\PY{l+s+s1}{\PYZsq{}}\PY{l+s+s1}{ENTRIESn\PYZus{}hourly}\PY{l+s+s1}{\PYZsq{}}\PY{p}{]} \PY{o}{=} \PY{n}{df}\PY{p}{[}\PY{l+s+s1}{\PYZsq{}}\PY{l+s+s1}{ENTRIESn}\PY{l+s+s1}{\PYZsq{}}\PY{p}{]} \PY{o}{\PYZhy{}} \PY{n}{df}\PY{p}{[}\PY{l+s+s1}{\PYZsq{}}\PY{l+s+s1}{ENTRIESn}\PY{l+s+s1}{\PYZsq{}}\PY{p}{]}\PY{o}{.}\PY{n}{shift}\PY{p}{(}\PY{l+m+mi}{1}\PY{p}{)}
             \PY{c+c1}{\PYZsh{}Substitui as linhas com valor == NaN por 1}
             \PY{n}{df}\PY{o}{.}\PY{n}{fillna}\PY{p}{(}\PY{l+m+mi}{1}\PY{p}{,} \PY{n}{inplace}\PY{o}{=}\PY{k+kc}{True}\PY{p}{)}
             \PY{c+c1}{\PYZsh{}converte o valor da coluna de float para int}
             \PY{n}{df}\PY{p}{[}\PY{l+s+s1}{\PYZsq{}}\PY{l+s+s1}{ENTRIESn\PYZus{}hourly}\PY{l+s+s1}{\PYZsq{}}\PY{p}{]} \PY{o}{=} \PY{n}{df}\PY{p}{[}\PY{l+s+s1}{\PYZsq{}}\PY{l+s+s1}{ENTRIESn\PYZus{}hourly}\PY{l+s+s1}{\PYZsq{}}\PY{p}{]}\PY{o}{.}\PY{n}{apply}\PY{p}{(}\PY{n}{np}\PY{o}{.}\PY{n}{int64}\PY{p}{)}
             \PY{k}{return} \PY{n}{df}
         
         \PY{n}{turnstile\PYZus{}regular} \PY{o}{=} \PY{n}{get\PYZus{}hourly\PYZus{}entries}\PY{p}{(}\PY{n}{turnstile\PYZus{}regular}\PY{p}{)}
         \PY{n}{turnstile\PYZus{}regular}\PY{o}{.}\PY{n}{head}\PY{p}{(}\PY{p}{)}
\end{Verbatim}


\begin{Verbatim}[commandchars=\\\{\}]
{\color{outcolor}Out[{\color{outcolor}42}]:}     C/A  UNIT       SCP STATION  LINENAME  DIVISION       DATEn     TIMEn  \textbackslash{}
         0  A002  R051  02-00-00   59 ST   NQR456W       BMT  05/27/2017  00:00:00   
         1  A002  R051  02-00-00   59 ST   NQR456W       BMT  05/27/2017  04:00:00   
         2  A002  R051  02-00-00   59 ST   NQR456W       BMT  05/27/2017  08:00:00   
         3  A002  R051  02-00-00   59 ST   NQR456W       BMT  05/27/2017  12:00:00   
         4  A002  R051  02-00-00   59 ST   NQR456W       BMT  05/27/2017  16:00:00   
         
              DESCn  ENTRIESn   EXITSn  ENTRIESn\_hourly  
         0  REGULAR   6195217  2098317                1  
         1  REGULAR   6195240  2098318               23  
         2  REGULAR   6195256  2098347               16  
         3  REGULAR   6195346  2098432               90  
         4  REGULAR   6195518  2098491              172  
\end{Verbatim}
            
    \subsubsection{\texorpdfstring{\emph{Exercicio
1.5}}{Exercicio 1.5}}\label{exercicio-1.5}

Faça o mesmo do exercicio anterior mas agora considerando as saidas,
coluna EXITSn. Para isso crie uma coluna chamada de EXITSn\_hourly e
insira a diferença entre a coluna EXITSn da linha atual versus a linha
anterior. Se tiver algum NaN, preencha/substitua por 0.

    \begin{Verbatim}[commandchars=\\\{\}]
{\color{incolor}In [{\color{incolor}46}]:} \PY{k+kn}{import} \PY{n+nn}{pandas}
         
         \PY{k}{def} \PY{n+nf}{get\PYZus{}hourly\PYZus{}exits}\PY{p}{(}\PY{n}{df}\PY{p}{)}\PY{p}{:}
             
             \PY{c+c1}{\PYZsh{}cria uma coluna nova (EXITSn\PYZus{}hourly) no dataframe e preenche suas linhas com a diferença de entradas}
             \PY{c+c1}{\PYZsh{} da linha atual com a linha anterior da coluna (EXITSn)}
             \PY{n}{df}\PY{p}{[}\PY{l+s+s1}{\PYZsq{}}\PY{l+s+s1}{EXITSn\PYZus{}hourly}\PY{l+s+s1}{\PYZsq{}}\PY{p}{]} \PY{o}{=} \PY{n}{df}\PY{p}{[}\PY{l+s+s1}{\PYZsq{}}\PY{l+s+s1}{EXITSn}\PY{l+s+s1}{\PYZsq{}}\PY{p}{]} \PY{o}{\PYZhy{}} \PY{n}{df}\PY{p}{[}\PY{l+s+s1}{\PYZsq{}}\PY{l+s+s1}{EXITSn}\PY{l+s+s1}{\PYZsq{}}\PY{p}{]}\PY{o}{.}\PY{n}{shift}\PY{p}{(}\PY{l+m+mi}{1}\PY{p}{)}
             \PY{c+c1}{\PYZsh{}Substitui as linhas com valor == NaN por 1}
             \PY{n}{df}\PY{o}{.}\PY{n}{fillna}\PY{p}{(}\PY{l+m+mi}{0}\PY{p}{,} \PY{n}{inplace}\PY{o}{=}\PY{k+kc}{True}\PY{p}{)}
             \PY{c+c1}{\PYZsh{}converte o valor da coluna de float para int}
             \PY{n}{df}\PY{p}{[}\PY{l+s+s1}{\PYZsq{}}\PY{l+s+s1}{EXITSn\PYZus{}hourly}\PY{l+s+s1}{\PYZsq{}}\PY{p}{]} \PY{o}{=} \PY{n}{df}\PY{p}{[}\PY{l+s+s1}{\PYZsq{}}\PY{l+s+s1}{EXITSn\PYZus{}hourly}\PY{l+s+s1}{\PYZsq{}}\PY{p}{]}\PY{o}{.}\PY{n}{apply}\PY{p}{(}\PY{n}{np}\PY{o}{.}\PY{n}{int64}\PY{p}{)}
             \PY{k}{return} \PY{n}{df}
         
         \PY{n}{turnstile\PYZus{}regular} \PY{o}{=} \PY{n}{get\PYZus{}hourly\PYZus{}exits}\PY{p}{(}\PY{n}{turnstile\PYZus{}regular}\PY{p}{)}
         \PY{n}{turnstile\PYZus{}regular}\PY{o}{.}\PY{n}{head}\PY{p}{(}\PY{p}{)}
\end{Verbatim}


\begin{Verbatim}[commandchars=\\\{\}]
{\color{outcolor}Out[{\color{outcolor}46}]:}     C/A  UNIT       SCP STATION  LINENAME  DIVISION       DATEn     TIMEn  \textbackslash{}
         0  A002  R051  02-00-00   59 ST   NQR456W       BMT  05/27/2017  00:00:00   
         1  A002  R051  02-00-00   59 ST   NQR456W       BMT  05/27/2017  04:00:00   
         2  A002  R051  02-00-00   59 ST   NQR456W       BMT  05/27/2017  08:00:00   
         3  A002  R051  02-00-00   59 ST   NQR456W       BMT  05/27/2017  12:00:00   
         4  A002  R051  02-00-00   59 ST   NQR456W       BMT  05/27/2017  16:00:00   
         
              DESCn  ENTRIESn   EXITSn  ENTRIESn\_hourly  EXITSn\_hourly  
         0  REGULAR   6195217  2098317                1              0  
         1  REGULAR   6195240  2098318               23              1  
         2  REGULAR   6195256  2098347               16             29  
         3  REGULAR   6195346  2098432               90             85  
         4  REGULAR   6195518  2098491              172             59  
\end{Verbatim}
            
    \subsubsection{\texorpdfstring{\emph{Exercicio
1.6}}{Exercicio 1.6}}\label{exercicio-1.6}

Dado uma variável de entrada que representa o tempo no formato de: ~~~~
"00:00:00" (hora: minutos: segundos) ~~~~ Escreva uma função para
extrair a parte da hora do tempo variável de entrada E devolva-o como um
número inteiro. Por exemplo: ~~~~~~~~ 1) se a hora for 00, seu código
deve retornar 0 ~~~~~~~~ 2) se a hora for 01, seu código deve retornar 1
~~~~~~~~ 3) se a hora for 21, seu código deve retornar 21 ~~~~~~~~ Por
favor, devolva a hora como um número inteiro.

    \begin{Verbatim}[commandchars=\\\{\}]
{\color{incolor}In [{\color{incolor}56}]:} \PY{k}{def} \PY{n+nf}{time\PYZus{}to\PYZus{}hour}\PY{p}{(}\PY{n}{time}\PY{p}{)}\PY{p}{:}
             
             \PY{n}{hour} \PY{o}{=} \PY{n}{time}\PY{o}{.}\PY{n}{split}\PY{p}{(}\PY{l+s+s1}{\PYZsq{}}\PY{l+s+s1}{:}\PY{l+s+s1}{\PYZsq{}}\PY{p}{)}\PY{p}{[}\PY{l+m+mi}{0}\PY{p}{]}
             \PY{k}{return} \PY{n+nb}{int}\PY{p}{(}\PY{n}{hour}\PY{p}{)}
         
         \PY{n}{time} \PY{o}{=} \PY{l+s+s1}{\PYZsq{}}\PY{l+s+s1}{00:20:00}\PY{l+s+s1}{\PYZsq{}}
         \PY{n+nb}{print}\PY{p}{(}\PY{n}{time\PYZus{}to\PYZus{}hour}\PY{p}{(}\PY{n}{time}\PY{p}{)}\PY{p}{)}
         \PY{n}{time} \PY{o}{=} \PY{l+s+s1}{\PYZsq{}}\PY{l+s+s1}{01:20:00}\PY{l+s+s1}{\PYZsq{}}
         \PY{n+nb}{print}\PY{p}{(}\PY{n}{time\PYZus{}to\PYZus{}hour}\PY{p}{(}\PY{n}{time}\PY{p}{)}\PY{p}{)}
         \PY{n}{time} \PY{o}{=} \PY{l+s+s1}{\PYZsq{}}\PY{l+s+s1}{21:20:00}\PY{l+s+s1}{\PYZsq{}}
         \PY{n+nb}{print}\PY{p}{(}\PY{n}{time\PYZus{}to\PYZus{}hour}\PY{p}{(}\PY{n}{time}\PY{p}{)}\PY{p}{)}
\end{Verbatim}


    \begin{Verbatim}[commandchars=\\\{\}]
0
1
21

    \end{Verbatim}

    \subsection{Exercicio 2 - Análise dos
dados}\label{exercicio-2---anuxe1lise-dos-dados}

\subsubsection{\texorpdfstring{\emph{Exercicio
2.1}}{Exercicio 2.1}}\label{exercicio-2.1}

Para verificar a relação entre o movimento do metrô e o clima,
precisaremos complementar os dados do arquivo já baixado com os dados do
clima. Nós complementamos para você este arquivo com os dados de clima
de Nova Iorque e disponibilizamos na área de materiais do projeto. Você
pode acessa-lo pelo link:
https://s3.amazonaws.com/content.udacity-data.com/courses/ud359/turnstile\_data\_master\_with\_weather.csv

Agora que temos nossos dados em um arquivo csv, escreva um código python
que leia este arquivo e salve-o em um data frame do pandas.

Dica:

Utilize o comando abaixo para ler o arquivo:

\begin{Shaded}
\begin{Highlighting}[]
\NormalTok{pd.read_csv(}\StringTok{'output_list.txt'}\NormalTok{, sep}\OperatorTok{=}\StringTok{","}\NormalTok{)}
\end{Highlighting}
\end{Shaded}

    \begin{Verbatim}[commandchars=\\\{\}]
{\color{incolor}In [{\color{incolor}3}]:} \PY{k+kn}{import} \PY{n+nn}{pandas} \PY{k}{as} \PY{n+nn}{pd}
        
        \PY{k}{def} \PY{n+nf}{read\PYZus{}turnstile\PYZus{}with\PYZus{}weather}\PY{p}{(}\PY{n}{filename}\PY{p}{)}\PY{p}{:}
            \PY{c+c1}{\PYZsh{}abre o arquivo como data frame do pandas}
            \PY{n}{turnstile\PYZus{}with\PYZus{}weather} \PY{o}{=} \PY{n}{pd}\PY{o}{.}\PY{n}{read\PYZus{}csv}\PY{p}{(}\PY{n}{filename}\PY{p}{)}
            \PY{k}{return} \PY{n}{turnstile\PYZus{}with\PYZus{}weather}
        
        \PY{n}{filename} \PY{o}{=} \PY{l+s+s2}{\PYZdq{}}\PY{l+s+s2}{turnstile\PYZus{}data\PYZus{}master\PYZus{}with\PYZus{}weather.csv}\PY{l+s+s2}{\PYZdq{}}
        \PY{n}{turnstile\PYZus{}with\PYZus{}weather} \PY{o}{=} \PY{n}{read\PYZus{}turnstile\PYZus{}with\PYZus{}weather}\PY{p}{(}\PY{n}{filename}\PY{p}{)}
        \PY{n}{turnstile\PYZus{}with\PYZus{}weather}\PY{o}{.}\PY{n}{head}\PY{p}{(}\PY{p}{)}
\end{Verbatim}


\begin{Verbatim}[commandchars=\\\{\}]
{\color{outcolor}Out[{\color{outcolor}3}]:}    Unnamed: 0  UNIT       DATEn     TIMEn  Hour    DESCn  ENTRIESn\_hourly  \textbackslash{}
        0           0  R001  2011-05-01  01:00:00     1  REGULAR              0.0   
        1           1  R001  2011-05-01  05:00:00     5  REGULAR            217.0   
        2           2  R001  2011-05-01  09:00:00     9  REGULAR            890.0   
        3           3  R001  2011-05-01  13:00:00    13  REGULAR           2451.0   
        4           4  R001  2011-05-01  17:00:00    17  REGULAR           4400.0   
        
           EXITSn\_hourly  maxpressurei  maxdewpti   {\ldots}     meandewpti  meanpressurei  \textbackslash{}
        0            0.0         30.31       42.0   {\ldots}           39.0          30.27   
        1          553.0         30.31       42.0   {\ldots}           39.0          30.27   
        2         1262.0         30.31       42.0   {\ldots}           39.0          30.27   
        3         3708.0         30.31       42.0   {\ldots}           39.0          30.27   
        4         2501.0         30.31       42.0   {\ldots}           39.0          30.27   
        
           fog  rain  meanwindspdi  mintempi  meantempi  maxtempi  precipi  thunder  
        0  0.0   0.0           5.0      50.0       60.0      69.0      0.0      0.0  
        1  0.0   0.0           5.0      50.0       60.0      69.0      0.0      0.0  
        2  0.0   0.0           5.0      50.0       60.0      69.0      0.0      0.0  
        3  0.0   0.0           5.0      50.0       60.0      69.0      0.0      0.0  
        4  0.0   0.0           5.0      50.0       60.0      69.0      0.0      0.0  
        
        [5 rows x 22 columns]
\end{Verbatim}
            
    \subsubsection{\texorpdfstring{\emph{Exercicio
2.2}}{Exercicio 2.2}}\label{exercicio-2.2}

Agora crie uma função que calcule a quantidade de dias chuvosos, para
isso retorne a contagem do numero de dias onde a coluna \emph{"rain"} é
igual a 1.

Dica: Você também pode achar que a interpretação de números como números
inteiros ou float pode não ~~~~ funcionar inicialmente. Para contornar
esta questão, pode ser útil converter ~~~~ esses números para números
inteiros. Isso pode ser feito escrevendo cast (coluna como inteiro).
~~~~ Então, por exemplo, se queríamos lançar a coluna maxtempi como um
número inteiro, nós devemos ~~~~ escrever algo como cast (maxtempi as
integer) = 76, em oposição a simplesmente ~~~~ onde maxtempi = 76.

    \begin{Verbatim}[commandchars=\\\{\}]
{\color{incolor}In [{\color{incolor}55}]:} \PY{k}{def} \PY{n+nf}{num\PYZus{}rainy\PYZus{}days}\PY{p}{(}\PY{n}{df}\PY{p}{)}\PY{p}{:}
             \PY{c+c1}{\PYZsh{}agrupa os dados pela data e soma a quantidade de itens encontrados}
             \PY{n}{df} \PY{o}{=} \PY{n}{df}\PY{o}{.}\PY{n}{groupby}\PY{p}{(}\PY{l+s+s1}{\PYZsq{}}\PY{l+s+s1}{DATEn}\PY{l+s+s1}{\PYZsq{}}\PY{p}{)}\PY{o}{.}\PY{n}{sum}\PY{p}{(}\PY{p}{)}
             \PY{c+c1}{\PYZsh{}print(df[\PYZsq{}rain\PYZsq{}])}
             \PY{c+c1}{\PYZsh{}conta o total de dias que choveu}
             \PY{k}{return} \PY{n}{df}\PY{p}{[}\PY{l+s+s1}{\PYZsq{}}\PY{l+s+s1}{rain}\PY{l+s+s1}{\PYZsq{}}\PY{p}{]}\PY{p}{[}\PY{n}{df}\PY{p}{[}\PY{l+s+s1}{\PYZsq{}}\PY{l+s+s1}{rain}\PY{l+s+s1}{\PYZsq{}}\PY{p}{]} \PY{o}{\PYZgt{}}\PY{o}{=} \PY{l+m+mi}{1}\PY{p}{]}\PY{o}{.}\PY{n}{count}\PY{p}{(}\PY{p}{)}
             
             
         \PY{n+nb}{print}\PY{p}{(}\PY{n}{num\PYZus{}rainy\PYZus{}days}\PY{p}{(}\PY{n}{turnstile\PYZus{}with\PYZus{}weather}\PY{p}{)}\PY{p}{)}
\end{Verbatim}


    \begin{Verbatim}[commandchars=\\\{\}]
10

    \end{Verbatim}

    \subsubsection{\texorpdfstring{\emph{Exercicio
2.3}}{Exercicio 2.3}}\label{exercicio-2.3}

Calcule se estava nebuloso ou não (0 ou 1) e a temperatura máxima para
fog (isto é, a temperatura máxima ~~~~ para dias nebulosos).

    \begin{Verbatim}[commandchars=\\\{\}]
{\color{incolor}In [{\color{incolor}162}]:} \PY{k}{def} \PY{n+nf}{max\PYZus{}temp\PYZus{}aggregate\PYZus{}by\PYZus{}fog}\PY{p}{(}\PY{n}{df}\PY{p}{)}\PY{p}{:}
              \PY{c+c1}{\PYZsh{}agrupa os dados pela data}
              \PY{n}{df} \PY{o}{=} \PY{n}{df}\PY{o}{.}\PY{n}{groupby}\PY{p}{(}\PY{l+s+s1}{\PYZsq{}}\PY{l+s+s1}{DATEn}\PY{l+s+s1}{\PYZsq{}}\PY{p}{)}\PY{o}{.}\PY{n}{max}\PY{p}{(}\PY{p}{)}
              \PY{c+c1}{\PYZsh{}recupera o total de dias que estava nebuloso}
              \PY{n}{fog\PYZus{}day} \PY{o}{=} \PY{n}{df}\PY{p}{[}\PY{l+s+s1}{\PYZsq{}}\PY{l+s+s1}{fog}\PY{l+s+s1}{\PYZsq{}}\PY{p}{]}\PY{p}{[}\PY{n}{df}\PY{p}{[}\PY{l+s+s1}{\PYZsq{}}\PY{l+s+s1}{fog}\PY{l+s+s1}{\PYZsq{}}\PY{p}{]} \PY{o}{\PYZgt{}}\PY{o}{=} \PY{l+m+mi}{1}\PY{p}{]}\PY{o}{.}\PY{n}{count}\PY{p}{(}\PY{p}{)}
              \PY{c+c1}{\PYZsh{}pega a maior temperatura de um dia nebuloso}
              \PY{n}{maxtempi\PYZus{}fog\PYZus{}day} \PY{o}{=} \PY{n}{df}\PY{p}{[}\PY{l+s+s1}{\PYZsq{}}\PY{l+s+s1}{maxtempi}\PY{l+s+s1}{\PYZsq{}}\PY{p}{]}\PY{p}{[}\PY{n}{df}\PY{p}{[}\PY{l+s+s1}{\PYZsq{}}\PY{l+s+s1}{fog}\PY{l+s+s1}{\PYZsq{}}\PY{p}{]} \PY{o}{\PYZgt{}}\PY{o}{=} \PY{l+m+mi}{1}\PY{p}{]}\PY{o}{.}\PY{n}{max}\PY{p}{(}\PY{p}{)}
              \PY{c+c1}{\PYZsh{}recupera o total de dias que não estava nebuloso}
              \PY{n}{non\PYZus{}fog\PYZus{}day} \PY{o}{=} \PY{n}{df}\PY{p}{[}\PY{l+s+s1}{\PYZsq{}}\PY{l+s+s1}{fog}\PY{l+s+s1}{\PYZsq{}}\PY{p}{]}\PY{p}{[}\PY{n}{df}\PY{p}{[}\PY{l+s+s1}{\PYZsq{}}\PY{l+s+s1}{fog}\PY{l+s+s1}{\PYZsq{}}\PY{p}{]} \PY{o}{==} \PY{l+m+mi}{0}\PY{p}{]}\PY{o}{.}\PY{n}{count}\PY{p}{(}\PY{p}{)}
              \PY{c+c1}{\PYZsh{}pega a maior temperatura de um dia que não está nebuloso}
              \PY{n}{maxtempi\PYZus{}non\PYZus{}fog\PYZus{}day} \PY{o}{=} \PY{n}{df}\PY{p}{[}\PY{l+s+s1}{\PYZsq{}}\PY{l+s+s1}{maxtempi}\PY{l+s+s1}{\PYZsq{}}\PY{p}{]}\PY{p}{[}\PY{n}{df}\PY{p}{[}\PY{l+s+s1}{\PYZsq{}}\PY{l+s+s1}{fog}\PY{l+s+s1}{\PYZsq{}}\PY{p}{]} \PY{o}{==} \PY{l+m+mi}{0}\PY{p}{]}\PY{o}{.}\PY{n}{max}\PY{p}{(}\PY{p}{)}
              \PY{k}{return} \PY{n}{fog\PYZus{}day}\PY{p}{,} \PY{n}{maxtempi\PYZus{}fog\PYZus{}day}\PY{p}{,} \PY{n}{non\PYZus{}fog\PYZus{}day}\PY{p}{,} \PY{n}{maxtempi\PYZus{}non\PYZus{}fog\PYZus{}day}
              
          
          \PY{n+nb}{print}\PY{p}{(}\PY{n}{max\PYZus{}temp\PYZus{}aggregate\PYZus{}by\PYZus{}fog}\PY{p}{(}\PY{n}{turnstile\PYZus{}with\PYZus{}weather}\PY{p}{)}\PY{p}{)}
\end{Verbatim}


    \begin{Verbatim}[commandchars=\\\{\}]
(5, 81.0, 25, 86.0)

    \end{Verbatim}

    \subsubsection{*Exercicio 2.4}\label{exercicio-2.4}

Calcule agora a média de 'meantempi' nos dias que são sábado ou domingo
(finais de semana):

    \begin{Verbatim}[commandchars=\\\{\}]
{\color{incolor}In [{\color{incolor}6}]:} \PY{k+kn}{import} \PY{n+nn}{csv}
        
        \PY{k+kn}{from} \PY{n+nn}{datetime} \PY{k}{import} \PY{n}{datetime}
        \PY{c+c1}{\PYZsh{}retorna um Date a partir de uma string no formado yyy\PYZhy{}MM\PYZhy{}dd}
        \PY{k}{def} \PY{n+nf}{to\PYZus{}date}\PY{p}{(}\PY{n}{dateStr}\PY{p}{)}\PY{p}{:}
            \PY{k}{return} \PY{n}{datetime}\PY{o}{.}\PY{n}{strptime}\PY{p}{(}\PY{n}{dateStr}\PY{p}{,} \PY{l+s+s1}{\PYZsq{}}\PY{l+s+s1}{\PYZpc{}}\PY{l+s+s1}{Y\PYZhy{}}\PY{l+s+s1}{\PYZpc{}}\PY{l+s+s1}{m\PYZhy{}}\PY{l+s+si}{\PYZpc{}d}\PY{l+s+s1}{\PYZsq{}}\PY{p}{)}
        
        \PY{c+c1}{\PYZsh{}retorna o dia da semana de uma dada}
        \PY{k}{def} \PY{n+nf}{get\PYZus{}weekday}\PY{p}{(}\PY{n}{date}\PY{p}{)}\PY{p}{:}   
            \PY{k}{return} \PY{n+nb}{int}\PY{p}{(}\PY{n}{date}\PY{o}{.}\PY{n}{strftime}\PY{p}{(}\PY{l+s+s1}{\PYZsq{}}\PY{l+s+s1}{\PYZpc{}}\PY{l+s+s1}{w}\PY{l+s+s1}{\PYZsq{}}\PY{p}{)}\PY{p}{)}
        
        \PY{c+c1}{\PYZsh{}verifica se a data é fim de semana}
        \PY{k}{def} \PY{n+nf}{is\PYZus{}weekend}\PY{p}{(}\PY{n}{weekday}\PY{p}{)}\PY{p}{:}
            \PY{k}{return} \PY{p}{(}\PY{n}{weekday} \PY{o}{==} \PY{l+m+mi}{0} \PY{o+ow}{or} \PY{n}{weekday} \PY{o}{==} \PY{l+m+mi}{6}\PY{p}{)}
        
        \PY{c+c1}{\PYZsh{}retorna a média da temperatura nos fins de semana}
        \PY{k}{def} \PY{n+nf}{avg\PYZus{}weekend\PYZus{}temperature}\PY{p}{(}\PY{n}{filename}\PY{p}{)}\PY{p}{:}
            \PY{n}{turnstile\PYZus{}with\PYZus{}weather} \PY{o}{=} \PY{n}{pd}\PY{o}{.}\PY{n}{read\PYZus{}csv}\PY{p}{(}\PY{n}{filename}\PY{p}{)}
            
            \PY{k}{with} \PY{n+nb}{open}\PY{p}{(}\PY{n}{filename}\PY{p}{,} \PY{l+s+s1}{\PYZsq{}}\PY{l+s+s1}{r}\PY{l+s+s1}{\PYZsq{}}\PY{p}{)} \PY{k}{as} \PY{n}{f\PYZus{}in}\PY{p}{:}
                \PY{n}{in\PYZus{}reader} \PY{o}{=} \PY{n}{csv}\PY{o}{.}\PY{n}{DictReader}\PY{p}{(}\PY{n}{f\PYZus{}in}\PY{p}{)}
                \PY{n}{total} \PY{o}{=} \PY{l+m+mi}{0}
                \PY{n}{count} \PY{o}{=} \PY{l+m+mi}{0}
                \PY{n}{previous\PYZus{}date} \PY{o}{=} \PY{k+kc}{None}
                \PY{k}{for} \PY{n}{row} \PY{o+ow}{in} \PY{n}{in\PYZus{}reader}\PY{p}{:}
                    \PY{n}{data} \PY{o}{=} \PY{n}{to\PYZus{}date}\PY{p}{(}\PY{n}{row}\PY{p}{[}\PY{l+s+s1}{\PYZsq{}}\PY{l+s+s1}{DATEn}\PY{l+s+s1}{\PYZsq{}}\PY{p}{]}\PY{p}{)}
                    \PY{k}{if} \PY{n}{is\PYZus{}weekend}\PY{p}{(}\PY{n}{get\PYZus{}weekday}\PY{p}{(}\PY{n}{data}\PY{p}{)}\PY{p}{)} \PY{o+ow}{and} \PY{p}{(}\PY{n}{data} \PY{o}{!=} \PY{n}{previous\PYZus{}date}\PY{p}{)}\PY{p}{:}
                        \PY{n}{total} \PY{o}{+}\PY{o}{=} \PY{n+nb}{float}\PY{p}{(}\PY{n}{row}\PY{p}{[}\PY{l+s+s1}{\PYZsq{}}\PY{l+s+s1}{meantempi}\PY{l+s+s1}{\PYZsq{}}\PY{p}{]}\PY{p}{)}
                        \PY{n}{count} \PY{o}{+}\PY{o}{=} \PY{l+m+mi}{1}
                    \PY{n}{previous\PYZus{}date} \PY{o}{=} \PY{n}{data}
                \PY{n+nb}{print}\PY{p}{(}\PY{n}{count}\PY{p}{)}
            \PY{k}{return} \PY{n}{total} \PY{o}{/} \PY{n}{count}
        
        \PY{n}{filename} \PY{o}{=} \PY{l+s+s2}{\PYZdq{}}\PY{l+s+s2}{turnstile\PYZus{}data\PYZus{}master\PYZus{}with\PYZus{}weather.csv}\PY{l+s+s2}{\PYZdq{}}
        \PY{n}{avg\PYZus{}weekend\PYZus{}temperature}\PY{p}{(}\PY{n}{filename}\PY{p}{)}
\end{Verbatim}


    \begin{Verbatim}[commandchars=\\\{\}]
9

    \end{Verbatim}

\begin{Verbatim}[commandchars=\\\{\}]
{\color{outcolor}Out[{\color{outcolor}6}]:} 65.11111111111111
\end{Verbatim}
            
    \subsubsection{*Exercicio 2.5}\label{exercicio-2.5}

Calcule a média da temperatura mínima 'mintempi' nos dias chuvosos onde
da temperatura mínima foi maior que do 55 graus:

    \begin{Verbatim}[commandchars=\\\{\}]
{\color{incolor}In [{\color{incolor}7}]:} \PY{c+c1}{\PYZsh{} Verifica se é um dia chuvoso}
        \PY{k}{def} \PY{n+nf}{is\PYZus{}rainy\PYZus{}day}\PY{p}{(}\PY{n}{rain}\PY{p}{)}\PY{p}{:}
            \PY{k}{return} \PY{n+nb}{int}\PY{p}{(}\PY{n+nb}{float}\PY{p}{(}\PY{n}{rain}\PY{p}{)}\PY{p}{)} \PY{o}{\PYZgt{}}\PY{o}{=} \PY{l+m+mi}{1}
        
        \PY{k}{def} \PY{n+nf}{avg\PYZus{}min\PYZus{}temperature}\PY{p}{(}\PY{n}{filename}\PY{p}{)}\PY{p}{:}
            \PY{n}{turnstile\PYZus{}with\PYZus{}weather} \PY{o}{=} \PY{n}{pd}\PY{o}{.}\PY{n}{read\PYZus{}csv}\PY{p}{(}\PY{n}{filename}\PY{p}{)}
            \PY{k}{with} \PY{n+nb}{open}\PY{p}{(}\PY{n}{filename}\PY{p}{,} \PY{l+s+s1}{\PYZsq{}}\PY{l+s+s1}{r}\PY{l+s+s1}{\PYZsq{}}\PY{p}{)} \PY{k}{as} \PY{n}{f\PYZus{}in}\PY{p}{:}
                \PY{n}{in\PYZus{}reader} \PY{o}{=} \PY{n}{csv}\PY{o}{.}\PY{n}{DictReader}\PY{p}{(}\PY{n}{f\PYZus{}in}\PY{p}{)}
                \PY{n}{total} \PY{o}{=} \PY{l+m+mi}{0}
                \PY{n}{count} \PY{o}{=} \PY{l+m+mi}{0}
                \PY{n}{previous\PYZus{}date} \PY{o}{=} \PY{k+kc}{None}
                \PY{k}{for} \PY{n}{row} \PY{o+ow}{in} \PY{n}{in\PYZus{}reader}\PY{p}{:}
                    \PY{n}{data} \PY{o}{=} \PY{n}{to\PYZus{}date}\PY{p}{(}\PY{n}{row}\PY{p}{[}\PY{l+s+s1}{\PYZsq{}}\PY{l+s+s1}{DATEn}\PY{l+s+s1}{\PYZsq{}}\PY{p}{]}\PY{p}{)}
                    \PY{n}{temp} \PY{o}{=} \PY{n+nb}{float}\PY{p}{(}\PY{n}{row}\PY{p}{[}\PY{l+s+s1}{\PYZsq{}}\PY{l+s+s1}{mintempi}\PY{l+s+s1}{\PYZsq{}}\PY{p}{]}\PY{p}{)}
                    \PY{k}{if} \PY{n}{is\PYZus{}rainy\PYZus{}day}\PY{p}{(}\PY{n}{row}\PY{p}{[}\PY{l+s+s1}{\PYZsq{}}\PY{l+s+s1}{rain}\PY{l+s+s1}{\PYZsq{}}\PY{p}{]}\PY{p}{)} \PY{o+ow}{and} \PY{n}{temp} \PY{o}{\PYZgt{}} \PY{l+m+mi}{55} \PY{o+ow}{and} \PY{p}{(}\PY{n}{data} \PY{o}{!=} \PY{n}{previous\PYZus{}date}\PY{p}{)}\PY{p}{:}
                        \PY{n}{total} \PY{o}{+}\PY{o}{=} \PY{n}{temp}
                        \PY{n}{count} \PY{o}{+}\PY{o}{=} \PY{l+m+mi}{1}
                    \PY{n}{previous\PYZus{}date} \PY{o}{=} \PY{n}{data}
                \PY{n+nb}{print}\PY{p}{(}\PY{n}{count}\PY{p}{)}
            \PY{k}{return} \PY{n}{total} \PY{o}{/} \PY{n}{count}
        
        \PY{n}{filename} \PY{o}{=} \PY{l+s+s2}{\PYZdq{}}\PY{l+s+s2}{turnstile\PYZus{}data\PYZus{}master\PYZus{}with\PYZus{}weather.csv}\PY{l+s+s2}{\PYZdq{}}
        \PY{n}{avg\PYZus{}min\PYZus{}temperature}\PY{p}{(}\PY{n}{filename}\PY{p}{)}
\end{Verbatim}


    \begin{Verbatim}[commandchars=\\\{\}]
4

    \end{Verbatim}

\begin{Verbatim}[commandchars=\\\{\}]
{\color{outcolor}Out[{\color{outcolor}7}]:} 61.25
\end{Verbatim}
            
    \subsubsection{*Exercicio 2.6}\label{exercicio-2.6}

Antes de realizar qualquer análise, pode ser útil olhar para os dados
que esperamos analisar. Mais especificamente, vamos examinR as entradas
por hora em nossos dados do metrô de Nova York para determinar a
distribuição dos dados. Estes dados são armazenados na coluna
{[}'ENTRIESn\_hourly'{]}. ~~~~ Trace dois histogramas nos mesmos eixos
para mostrar as entradas quando esta chovendo vs quando não está
chovendo. Abaixo está um exemplo sobre como traçar histogramas com
pandas e matplotlib: ~~~~

\begin{Shaded}
\begin{Highlighting}[]
\NormalTok{Turnstile_weather [}\StringTok{'column_to_graph'}\NormalTok{]. Hist ()}
\end{Highlighting}
\end{Shaded}

    \begin{Verbatim}[commandchars=\\\{\}]
{\color{incolor}In [{\color{incolor}75}]:} \PY{k+kn}{import} \PY{n+nn}{numpy} \PY{k}{as} \PY{n+nn}{np}
         \PY{k+kn}{import} \PY{n+nn}{pandas}
         \PY{k+kn}{import} \PY{n+nn}{matplotlib}\PY{n+nn}{.}\PY{n+nn}{pyplot} \PY{k}{as} \PY{n+nn}{plt}
         
         \PY{k}{def} \PY{n+nf}{entries\PYZus{}histogram}\PY{p}{(}\PY{n}{turnstile\PYZus{}weather}\PY{p}{)}\PY{p}{:}
             
             \PY{n}{plt}\PY{o}{.}\PY{n}{figure}\PY{p}{(}\PY{p}{)}
             \PY{n}{entries\PYZus{}hourly\PYZus{}rainy\PYZus{}day} \PY{o}{=} \PY{n}{turnstile\PYZus{}weather}\PY{p}{[}\PY{l+s+s1}{\PYZsq{}}\PY{l+s+s1}{ENTRIESn\PYZus{}hourly}\PY{l+s+s1}{\PYZsq{}}\PY{p}{]}\PY{p}{[}\PY{n}{turnstile\PYZus{}weather}\PY{p}{[}\PY{l+s+s1}{\PYZsq{}}\PY{l+s+s1}{rain}\PY{l+s+s1}{\PYZsq{}}\PY{p}{]} \PY{o}{\PYZgt{}}\PY{o}{=} \PY{l+m+mi}{1}\PY{p}{]}
             \PY{n}{entries\PYZus{}hourly\PYZus{}non\PYZus{}rainy\PYZus{}day} \PY{o}{=} \PY{n}{turnstile\PYZus{}weather}\PY{p}{[}\PY{l+s+s1}{\PYZsq{}}\PY{l+s+s1}{ENTRIESn\PYZus{}hourly}\PY{l+s+s1}{\PYZsq{}}\PY{p}{]}\PY{p}{[}\PY{n}{turnstile\PYZus{}weather}\PY{p}{[}\PY{l+s+s1}{\PYZsq{}}\PY{l+s+s1}{rain}\PY{l+s+s1}{\PYZsq{}}\PY{p}{]} \PY{o}{==} \PY{l+m+mi}{0}\PY{p}{]}
             \PY{n}{plt}\PY{o}{.}\PY{n}{hist}\PY{p}{(}\PY{n}{entries\PYZus{}hourly\PYZus{}rainy\PYZus{}day}\PY{p}{,} \PY{n}{color}\PY{o}{=}\PY{l+s+s1}{\PYZsq{}}\PY{l+s+s1}{blue}\PY{l+s+s1}{\PYZsq{}}\PY{p}{,} \PY{n}{bins}\PY{o}{=}\PY{l+m+mi}{50}\PY{p}{,} \PY{n}{alpha}\PY{o}{=}\PY{l+m+mi}{1}\PY{p}{,} \PY{n}{label}\PY{o}{=}\PY{l+s+s1}{\PYZsq{}}\PY{l+s+s1}{Chovendo}\PY{l+s+s1}{\PYZsq{}}\PY{p}{)}
             \PY{n}{plt}\PY{o}{.}\PY{n}{hist}\PY{p}{(}\PY{n}{entries\PYZus{}hourly\PYZus{}non\PYZus{}rainy\PYZus{}day}\PY{p}{,}  \PY{n}{color}\PY{o}{=}\PY{l+s+s1}{\PYZsq{}}\PY{l+s+s1}{green}\PY{l+s+s1}{\PYZsq{}}\PY{p}{,} \PY{n}{bins}\PY{o}{=}\PY{l+m+mi}{50}\PY{p}{,} \PY{n}{alpha}\PY{o}{=}\PY{l+m+mf}{0.2}\PY{p}{,} \PY{n}{label}\PY{o}{=}\PY{l+s+s1}{\PYZsq{}}\PY{l+s+s1}{Sem chover}\PY{l+s+s1}{\PYZsq{}}\PY{p}{)}
             \PY{n}{plt}\PY{o}{.}\PY{n}{title}\PY{p}{(}\PY{l+s+s1}{\PYZsq{}}\PY{l+s+s1}{Entradas por hora de acordo com o tempo}\PY{l+s+s1}{\PYZsq{}}\PY{p}{)}
             \PY{n}{plt}\PY{o}{.}\PY{n}{xlabel}\PY{p}{(}\PY{l+s+s1}{\PYZsq{}}\PY{l+s+s1}{Número de entradas}\PY{l+s+s1}{\PYZsq{}}\PY{p}{)}
             \PY{n}{plt}\PY{o}{.}\PY{n}{ylabel}\PY{p}{(}\PY{l+s+s1}{\PYZsq{}}\PY{l+s+s1}{Frequência das entradas}\PY{l+s+s1}{\PYZsq{}}\PY{p}{)}
             \PY{n}{plt}\PY{o}{.}\PY{n}{axis}\PY{p}{(}\PY{p}{[}\PY{l+m+mi}{0}\PY{p}{,} \PY{l+m+mi}{10000}\PY{p}{,} \PY{l+m+mi}{0}\PY{p}{,} \PY{l+m+mi}{100000}\PY{p}{]}\PY{p}{)}
             \PY{n}{plt}\PY{o}{.}\PY{n}{legend}\PY{p}{(}\PY{p}{)}
             
             \PY{k}{return} \PY{n}{plt}
         
         \PY{n}{entries\PYZus{}histogram}\PY{p}{(}\PY{n}{turnstile\PYZus{}with\PYZus{}weather}\PY{p}{)}
         \PY{n}{plt}\PY{o}{.}\PY{n}{show}\PY{p}{(}\PY{p}{)}
\end{Verbatim}


    \begin{center}
    \adjustimage{max size={0.9\linewidth}{0.9\paperheight}}{output_26_0.png}
    \end{center}
    { \hspace*{\fill} \\}
    
    \subsubsection{*Exercicio 2.7}\label{exercicio-2.7}

Os dados que acabou de plotar que tipo de ditribuição? Existe diferença
na distribuição entre dias chuvosos e não chuvosos?

    ** Resposta **:

A distribuição é assimétrica negativa devido os valores estarem
concentrados à esquerda.

Sim. Podemos visualizar que os dias com chuva possuem menor entradas por
hora comparado aos sem chuva, mas devemos levar em conta a quantidade de
dias com registro de chuva.

    \subsubsection{*Exercicio 2.8}\label{exercicio-2.8}

Construa uma função que que retorne:

\begin{enumerate}
\def\labelenumi{\arabic{enumi}.}
\tightlist
\item
  A média das entradas com chuva
\item
  A média das entradas sem chuva
\end{enumerate}

    \begin{Verbatim}[commandchars=\\\{\}]
{\color{incolor}In [{\color{incolor}78}]:} \PY{k+kn}{import} \PY{n+nn}{numpy} \PY{k}{as} \PY{n+nn}{np}
         
         \PY{k+kn}{import} \PY{n+nn}{pandas}
         
         \PY{k}{def} \PY{n+nf}{means}\PY{p}{(}\PY{n}{turnstile\PYZus{}weather}\PY{p}{)}\PY{p}{:}
             \PY{n}{with\PYZus{}rain\PYZus{}mean} \PY{o}{=} \PY{n}{turnstile\PYZus{}weather}\PY{p}{[}\PY{l+s+s1}{\PYZsq{}}\PY{l+s+s1}{ENTRIESn\PYZus{}hourly}\PY{l+s+s1}{\PYZsq{}}\PY{p}{]}\PY{p}{[}\PY{n}{turnstile\PYZus{}weather}\PY{p}{[}\PY{l+s+s1}{\PYZsq{}}\PY{l+s+s1}{rain}\PY{l+s+s1}{\PYZsq{}}\PY{p}{]} \PY{o}{\PYZgt{}}\PY{o}{=} \PY{l+m+mi}{1}\PY{p}{]}\PY{o}{.}\PY{n}{mean}\PY{p}{(}\PY{p}{)}
             \PY{n}{without\PYZus{}rain\PYZus{}mean} \PY{o}{=} \PY{n}{turnstile\PYZus{}weather}\PY{p}{[}\PY{l+s+s1}{\PYZsq{}}\PY{l+s+s1}{ENTRIESn\PYZus{}hourly}\PY{l+s+s1}{\PYZsq{}}\PY{p}{]}\PY{p}{[}\PY{n}{turnstile\PYZus{}weather}\PY{p}{[}\PY{l+s+s1}{\PYZsq{}}\PY{l+s+s1}{rain}\PY{l+s+s1}{\PYZsq{}}\PY{p}{]} \PY{o}{==} \PY{l+m+mi}{0}\PY{p}{]}\PY{o}{.}\PY{n}{mean}\PY{p}{(}\PY{p}{)}
             \PY{k}{return} \PY{n}{with\PYZus{}rain\PYZus{}mean}\PY{p}{,} \PY{n}{without\PYZus{}rain\PYZus{}mean}
         
         \PY{n+nb}{print}\PY{p}{(}\PY{n}{means}\PY{p}{(}\PY{n}{turnstile\PYZus{}with\PYZus{}weather}\PY{p}{)}\PY{p}{)}
\end{Verbatim}


    \begin{Verbatim}[commandchars=\\\{\}]
(1105.4463767458733, 1090.278780151855)

    \end{Verbatim}

    Responda as perguntas abaixo de acordo com a saida das suas funções:

\begin{enumerate}
\def\labelenumi{\arabic{enumi}.}
\tightlist
\item
  Qual a média das entradas com chuva? 1105.45
\item
  Qual a média das entradas sem chuva? 1090.28
\end{enumerate}

    ** Resposta **: Substitua este texto pela sua resposta!

    \subsection{Exercicio 3 - Map Reduce}\label{exercicio-3---map-reduce}

\subsubsection{\texorpdfstring{\emph{Exercicio
3.1}}{Exercicio 3.1}}\label{exercicio-3.1}

A entrada para esse exercício e o mesmo arquivo da seção anterior
(Exercicio 2). Você pode baixar o arquivo neste link:

https://s3.amazonaws.com/content.udacity-data.com/courses/ud359/turnstile\_data\_master\_with\_weather.csv

Varmos criar um mapeador agora. Para cada linha de entrada, a saída do
mapeador deve IMPRIMIR (não retornar) a UNIT como uma chave e o número
de ENTRIESn\_hourly como o valor. Separe a chave e o valor por uma guia.
Por exemplo: 'R002 ~t105105.0'

Exporte seu mapeador em um arquivo chamado mapper\_result.txt e envie
esse arquivo juntamente com a sua submissão. O código para exportar seu
mapeador já está escrito no código abaixo.

    \begin{Verbatim}[commandchars=\\\{\}]
{\color{incolor}In [{\color{incolor}3}]:} \PY{k+kn}{import} \PY{n+nn}{sys}
        
        \PY{k}{def} \PY{n+nf}{mapper}\PY{p}{(}\PY{p}{)}\PY{p}{:}
            \PY{k}{for} \PY{n}{line} \PY{o+ow}{in} \PY{n}{sys}\PY{o}{.}\PY{n}{stdin}\PY{p}{:}
                \PY{k}{try}\PY{p}{:}
                    \PY{n}{data} \PY{o}{=} \PY{n}{line}\PY{o}{.}\PY{n}{strip}\PY{p}{(}\PY{p}{)}\PY{o}{.}\PY{n}{split}\PY{p}{(}\PY{l+s+s1}{\PYZsq{}}\PY{l+s+s1}{,}\PY{l+s+s1}{\PYZsq{}}\PY{p}{)}
                    \PY{k}{if} \PY{n+nb}{len}\PY{p}{(}\PY{n}{data}\PY{p}{)} \PY{o}{==} \PY{l+m+mi}{22}\PY{p}{:}
                        \PY{n}{colum}\PY{p}{,}\PY{n}{UNIT}\PY{p}{,}\PY{n}{DATEn}\PY{p}{,}\PY{n}{TIMEn}\PY{p}{,}\PY{n}{Hour}\PY{p}{,}\PY{n}{DESCn}\PY{p}{,}\PY{n}{ENTRIESn\PYZus{}hourly}\PY{p}{,}\PY{n}{EXITSn\PYZus{}hourly}\PY{p}{,}\PY{n}{maxpressurei}\PY{p}{,}\PY{n}{maxdewpti}\PY{p}{,}\PY{n}{mindewpti}\PY{p}{,}\PY{n}{minpressurei}\PY{p}{,}\PY{n}{meandewpti}\PY{p}{,}\PY{n}{meanpressurei}\PY{p}{,}\PY{n}{fog}\PY{p}{,}\PY{n}{rain}\PY{p}{,}\PY{n}{meanwindspdi}\PY{p}{,}\PY{n}{mintempi}\PY{p}{,}\PY{n}{meantempi}\PY{p}{,}\PY{n}{maxtempi}\PY{p}{,}\PY{n}{precipi}\PY{p}{,}\PY{n}{thunder} \PY{o}{=} \PY{n}{data}    
                        \PY{n+nb}{print}\PY{p}{(}\PY{n}{UNIT}\PY{p}{,} \PY{n}{ENTRIESn\PYZus{}hourly}\PY{p}{)}
                \PY{k}{except} \PY{n+ne}{ValueError}\PY{p}{:}
                    \PY{k}{continue}    
        \PY{k}{if} \PY{n+nv+vm}{\PYZus{}\PYZus{}name\PYZus{}\PYZus{}} \PY{o}{==} \PY{l+s+s2}{\PYZdq{}}\PY{l+s+s2}{\PYZus{}\PYZus{}main\PYZus{}\PYZus{}}\PY{l+s+s2}{\PYZdq{}}\PY{p}{:}
            \PY{n}{sys}\PY{o}{.}\PY{n}{stdin} \PY{o}{=} \PY{n+nb}{open}\PY{p}{(}\PY{l+s+s1}{\PYZsq{}}\PY{l+s+s1}{turnstile\PYZus{}data\PYZus{}master\PYZus{}with\PYZus{}weather.csv}\PY{l+s+s1}{\PYZsq{}}\PY{p}{)}
            \PY{n}{sys}\PY{o}{.}\PY{n}{stdout} \PY{o}{=} \PY{n+nb}{open}\PY{p}{(}\PY{l+s+s1}{\PYZsq{}}\PY{l+s+s1}{mapper\PYZus{}result.txt}\PY{l+s+s1}{\PYZsq{}}\PY{p}{,} \PY{l+s+s1}{\PYZsq{}}\PY{l+s+s1}{w}\PY{l+s+s1}{\PYZsq{}}\PY{p}{)}
            \PY{n}{mapper}\PY{p}{(}\PY{p}{)}
\end{Verbatim}


    \subsubsection{\texorpdfstring{\emph{Exercicio
3.2}}{Exercicio 3.2}}\label{exercicio-3.2}

Agora crie o redutor. Dado o resultado do mapeador do exercicio
anterior, o redutor deve imprimir(Não retornar) uma linha por UNIT,
juntamente com o número total de ENTRIESn\_hourly.Ao longo de maio (que
é a duração dos nossos dados), separados por uma guia. Um exemplo de
linha de saída do redutor pode ser assim: 'R001 ~t500625.0'

Você pode assumir que a entrada para o redutor está ordenada de tal
forma que todas as linhas correspondentes a uma unidade particular são
agrupados. No entanto a saida do redutor terá repetição pois existem
lojas que aparecem em locais diferentes dos arquivos.

Exporte seu redutor em um arquivo chamado reducer\_result.txt e envie
esse arquivo juntamente com a sua submissão.

    \begin{Verbatim}[commandchars=\\\{\}]
{\color{incolor}In [{\color{incolor}20}]:} \PY{k}{def} \PY{n+nf}{reducer}\PY{p}{(}\PY{p}{)}\PY{p}{:}
             \PY{n}{oldKey} \PY{o}{=} \PY{k+kc}{None}
             \PY{n}{total} \PY{o}{=} \PY{l+m+mi}{0}
             \PY{k}{for} \PY{n}{line} \PY{o+ow}{in} \PY{n}{sys}\PY{o}{.}\PY{n}{stdin}\PY{p}{:}
                 \PY{n}{data\PYZus{}mapped} \PY{o}{=} \PY{n}{line}\PY{o}{.}\PY{n}{strip}\PY{p}{(}\PY{p}{)}\PY{o}{.}\PY{n}{split}\PY{p}{(}\PY{l+s+s1}{\PYZsq{}}\PY{l+s+s1}{ }\PY{l+s+s1}{\PYZsq{}}\PY{p}{)}
                 \PY{k}{if} \PY{n+nb}{len}\PY{p}{(}\PY{n}{data\PYZus{}mapped}\PY{p}{)} \PY{o}{!=} \PY{l+m+mi}{2}\PY{p}{:}
                     \PY{k}{continue}
                 \PY{n}{unit}\PY{p}{,} \PY{n}{entries} \PY{o}{=} \PY{n}{data\PYZus{}mapped}
                 \PY{k}{if} \PY{l+s+s1}{\PYZsq{}}\PY{l+s+s1}{UNIT}\PY{l+s+s1}{\PYZsq{}} \PY{o}{==} \PY{n}{unit}\PY{p}{:}
                     \PY{k}{continue}
                 
                 \PY{k}{if} \PY{n}{oldKey} \PY{o+ow}{and} \PY{n}{oldKey} \PY{o}{!=} \PY{n}{unit}\PY{p}{:}
                     \PY{n+nb}{print}\PY{p}{(}\PY{n}{oldKey}\PY{p}{,} \PY{l+s+s2}{\PYZdq{}}\PY{l+s+se}{\PYZbs{}t}\PY{l+s+s2}{\PYZdq{}}\PY{p}{,} \PY{n}{total}\PY{p}{)}
                     \PY{n}{total} \PY{o}{=} \PY{l+m+mi}{0}
                     \PY{n}{oldKey} \PY{o}{=} \PY{n}{unit}
                 
                 \PY{n}{total} \PY{o}{+}\PY{o}{=} \PY{n+nb}{float}\PY{p}{(}\PY{n}{entries}\PY{p}{)}
                 \PY{n}{oldKey} \PY{o}{=} \PY{n}{unit}
                 
             \PY{k}{if} \PY{n}{oldKey} \PY{o}{!=} \PY{k+kc}{None}\PY{p}{:}
                 \PY{n+nb}{print}\PY{p}{(}\PY{n}{oldKey}\PY{p}{,} \PY{l+s+s2}{\PYZdq{}}\PY{l+s+se}{\PYZbs{}t}\PY{l+s+s2}{\PYZdq{}}\PY{p}{,} \PY{n}{total}\PY{p}{)}
         
         
         \PY{k}{if} \PY{n+nv+vm}{\PYZus{}\PYZus{}name\PYZus{}\PYZus{}} \PY{o}{==} \PY{l+s+s2}{\PYZdq{}}\PY{l+s+s2}{\PYZus{}\PYZus{}main\PYZus{}\PYZus{}}\PY{l+s+s2}{\PYZdq{}}\PY{p}{:}     
             \PY{n}{sys}\PY{o}{.}\PY{n}{stdin} \PY{o}{=} \PY{n+nb}{open}\PY{p}{(}\PY{l+s+s1}{\PYZsq{}}\PY{l+s+s1}{mapper\PYZus{}result.txt}\PY{l+s+s1}{\PYZsq{}}\PY{p}{)}
             \PY{n}{sys}\PY{o}{.}\PY{n}{stdout} \PY{o}{=} \PY{n+nb}{open}\PY{p}{(}\PY{l+s+s1}{\PYZsq{}}\PY{l+s+s1}{reducer\PYZus{}result.txt}\PY{l+s+s1}{\PYZsq{}}\PY{p}{,} \PY{l+s+s1}{\PYZsq{}}\PY{l+s+s1}{w}\PY{l+s+s1}{\PYZsq{}}\PY{p}{)}
             \PY{n}{reducer}\PY{p}{(}\PY{p}{)}
\end{Verbatim}



    % Add a bibliography block to the postdoc
    
    
    
    \end{document}
